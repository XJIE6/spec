%% Простая презентация с примером включения программного кода и
%% пошаговых спецэффектов
\documentclass{beamer}
\usepackage{fontspec}
\usepackage{xunicode}
\usepackage{xltxtra}
\usepackage{xecyr}
\usepackage{hyperref}
\usepackage{ stmaryrd }
\usepackage{tikz}
\setmainfont[Mapping=tex-text]{DejaVu Serif}
\setsansfont[Mapping=tex-text]{DejaVu Sans}
\setmonofont[Mapping=tex-text]{DejaVu Sans Mono}
\usepackage{polyglossia}
\setdefaultlanguage{russian}
\usepackage{graphicx}
\usepackage{listings}
\lstdefinestyle{mycode}{
  belowcaptionskip=1\baselineskip,
  breaklines=true,
  xleftmargin=\parindent,
  showstringspaces=false,
  basicstyle=\footnotesize\ttfamily,
  keywordstyle=\bfseries,
  commentstyle=\itshape\color{gray!40!black},
  stringstyle=\color{red},
  numbers=left,
  numbersep=5pt,
  numberstyle=\tiny\color{gray},
}
\lstset{escapechar=@,style=mycode}

\begin{document}
\title{Самоприменимый специализатор для машинного кода}
\subtitle{предварительные результаты}
\author{Юрий Кравченко\\{\footnotesize\textcolor{gray}{руководитель Березун Даниил Андреевич}}}
\institute{СПбАУ}
\frame{\titlepage}

\begin{frame}\frametitle{Специализатор}
\begin{block}{Обозначения}
$\llbracket p \rrbracket_L [a_1, a_2, \dots, a_n] = b$
\end{block}
\pause
\begin{block}{Определение}
Программу $spec$ назовём специализатором, если\\
$\llbracket p \rrbracket_{L_1} [d_1, d_2] = d_3$\\
$\llbracket spec \rrbracket_{L_2} [p, d_1] = q$\\
$\llbracket q \rrbracket_{L_1} [d_2] = d_3$\\
\end{block}
\end{frame}

\begin{frame}\frametitle{Третья проеция Футамуры}
\begin{block}{Утверждение}
$\llbracket spec \rrbracket_L [spec, spec] = compgen$ \\
$\llbracket compgen \rrbracket_L [interpreter] = compiler$\\
\end{block}

\end{frame}


\begin{frame}\frametitle{В чём подвох?}
\begin{itemize}
    \item<2-> {Тривиальный spec}
    \item<3-> {Неразрешимость}
    \item<4-> {Сложно реализовать}
\end{itemize}
\end{frame}

\begin{frame}\frametitle{Идея}
\begin{itemize}
    \item<2-> {Специализатор для машинного кода}
    \item<3-> {Специализатор на СИ}
    \item<4-> {По аналогии со специализатором для Flow Chart}
\end{itemize}
\end{frame}

\begin{frame}\frametitle{Процесс разработки}
\begin{itemize}
    \item<2->  {Несколько версий архитектуры}
    \item<3-> {Поэтапная разработка}
    \item<4-> {Реализованы самые популярные операторы}
\end{itemize}
\end{frame}

\lstset{language=C}
\begin{frame}[fragile]\frametitle{Интерпретация}
\begin{lstlisting}
void sort(int len, int* a) {
    for (int i = 0; i < len; ++i) {
        for (int j = i + 1; j < len; ++j) {
            if (a[i] > a[j]) {
                int k = a[i];
                a[i] = a[j];
                a[j] = k;
            }
        }
    }
}
\end{lstlisting}
\end{frame}

\lstset{language=C}
\begin{frame}[fragile]\frametitle{Работа с памятью}
\begin{lstlisting}
char* eratosphen(int n) {
    char* a = my_malloc(n);
    for (int i = 0; i < n; ++i) {
        a[i] = 1;
    }
    a[0] = 0;
    a[1] = 0;
    for (int i = 2; i < n; ++i) {
        if (a[i]) {
            for (int j = i * i; j < n; j += i) {
                a[j] = 0;
            }
        }
    }
    return a;
}
\end{lstlisting}
\end{frame}

\lstset{language=C}
\begin{frame}[fragile]\frametitle{Простая специализация}
\begin{lstlisting}
int dict(int len, int* a, int* b, int c) {
    for (int i = 0; i < len; ++i) {
        if (a[i] == c) {
            if (b[i] == 0) {
                return 1;
            }
            else {
                return 2;
            }
        }
    }
    return 3;
}
\end{lstlisting}
\end{frame}

\lstset{language=C}
\begin{frame}[fragile]\frametitle{Полный цикл обработки}
\begin{lstlisting}
int pow(int a, int b) {
    if (b == 0) {
        return 1;
    }
    return my_pow(a, b - 1) * a;
}
\end{lstlisting}
\end{frame}

\begin{frame}\frametitle{KMP тест}
\begin{tikzpicture}
\LARGE
\node [anchor=west] at (.1, 2.5) {Coming soon};
\draw [fill=gray] (0,0) rectangle (10, 2);
\draw [fill={rgb:red,1;green,2;blue,3}] (0,0) rectangle (9, 2);
\end{tikzpicture}
\end{frame}

\begin{frame}\frametitle{Конец}
https://github.com/XJIE6/spec
\end{frame}
\end{document}