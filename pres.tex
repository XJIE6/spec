\documentclass[demo]{beamer}
\usepackage{amsmath}
\usepackage{fontspec}
\usepackage{xunicode}
\usepackage{xltxtra}
\usepackage{xecyr}
\usepackage{hyperref}
\usepackage{ stmaryrd }
\usepackage{tikz}
\setmainfont[Mapping=tex-text]{DejaVu Serif}
\setsansfont[Mapping=tex-text]{DejaVu Sans}
\setmonofont[Mapping=tex-text]{DejaVu Sans Mono}
\usepackage{polyglossia}
\setdefaultlanguage{russian}
\usepackage{graphicx}
\definecolor{ao}{rgb}{0.0, 0.4, 0.0}
\addtobeamertemplate{navigation symbols}{}{%
    \usebeamerfont{footline}%
    \usebeamercolor[fg]{footline}%
    \hspace{1em}%
    \insertframenumber/\inserttotalframenumber
}
\usepackage{listings}
\lstdefinestyle{mycode}{
  belowcaptionskip=1\baselineskip,
  breaklines=true,
  xleftmargin=\parindent,
  showstringspaces=false,
  basicstyle=\footnotesize\ttfamily,
  keywordstyle=\bfseries,
  commentstyle=\itshape\color{gray!40!black},
  stringstyle=\color{ao},
  numbers=left,
  numbersep=5pt,
  numberstyle=\tiny\color{gray},
}
\lstset{escapechar=@,style=mycode}
\newcommand{\xdownarrow}[1]{%
  {\left\downarrow\vbox to #1{}\right.\kern-\nulldelimiterspace}
}

\begin{document}
\title{Специализация машинного кода}
\author{Юрий Кравченко\\{\footnotesize\textcolor{gray}{руководитель Березун Даниил Андреевич}}}
\institute{СПбАУ}
\frame{\titlepage}

\begin{frame}\frametitle{Специализация}
  \begin{block}{Традиционное исполнение программы}
    $$\llbracket p \rrbracket_L [in_1, in_2, \dots] = out$$
  \end{block}
  \begin{block}{Специализатор}
    Программу $spec$ назовём специализатором, если
    $$\begin{array}{l l c l}
        \llbracket spec \rrbracket_{L_2} & [p, in_1] &= &p_{spec}\\
        \llbracket p_{spec} \rrbracket_{L_1} & [in_2,\dots] &= &out
      \end{array}$$
    \end{block}
\end{frame}

\begin{frame}{Цель специализации}
  $source$ --- программа на языке S\\
  $int$ --- интерпретатор для языка S на языке L\\
  \vfill
  \begin{block}{Проекции Футамуры \hfill [1973]}
    $$
    \begin{array}{l l l c l}
      I & \llbracket spec \rrbracket_L &[int, source] &= &target\\
      II & \llbracket spec \rrbracket_L &[spec, int] &= &comp\\
      III & \llbracket spec \rrbracket_L &[spec, spec] &= &cogen
    \end{array}$$
  \end{block}
  \vfill
  \begin{block}{Вывод}
    $$Interpreter \overset{spec}{\longrightarrow} Compiler$$
  \end{block}
\end{frame}

\begin{frame}\frametitle{В чём подвох?}
  \begin{itemize}
  \item{ Компилятор в язык реализации интерпретатора
      $$Interpreter^{S}_{\mathbf{L}}
      \xrightarrow{\qquad spec^{\mathbf{L}}_{\mathbf{L}}\qquad} Compiler^{S \to \mathbf{L}}_{\mathbf{L}}$$
      \hfill {\large \color{red}Это основная проблема}
    }
    \vfill
  \item {Апостериорный факт: реализовывать специализаторы сложно ($\sim$ как компиляторы)}
  \end{itemize}
\end{frame}

\begin{frame}{Релевантные исследования}
  Все текущие исследования имеют одну из двух проблем
  \begin{itemize}
  \item Искусственный язык
    \vfill
  \item Нет возможности самоприменения
  \end{itemize}
  \vfill
  Partial Evaluation of Machine Code \hfill [2015]
    \begin{itemize}
    \item Подмножество IA-32
    \vfill
    \item Использование сторонней закрытой библиотеки
    \vfill
    \item \large Написан на Java $\Rightarrow$ нельзя самоприменить
  	\end{itemize}
\end{frame}

\begin{frame}\frametitle{Идея}
  \begin{itemize}
  \item {Специализатор для машинного кода
      $$Interpreter^{S}_{\mathbf{ASM}}
      \xrightarrow{\qquad spec^{ \mathbf{ASM}}_{\mathbf{ASM}}\qquad} Compiler^{S \to \mathbf{ASM}}$$
    }
    \vfill
  \item {Как получить $Interpreter^{S}_{\mathbf{ASM}}$?
      $$\llbracket gcc \rrbracket [Interpreter^{S}_{\textsc{C}}] =
      Interpreter^{S}_{\mathbf{ASM}}$$}
    \vfill
  \item {Как получить $spec^{\mathbf{ASM}}_{\mathbf{ASM}}$?
      $$\llbracket gcc \rrbracket [spec^{ \mathbf{ASM}}_{\textsc{C}}] =
      spec^{\mathbf{ASM}}_{\mathbf{ASM}}$$}
    \vfill
  \item{Как получить $spec^{\mathbf{ASM}}_{\textsc{C}}$?}
    \vfill
  \end{itemize}
\end{frame}

\begin{frame}\frametitle{Цель и задачи}
\begin{block}{Цель}
\begin{itemize}
\item Исследование возможностей специализатора машинного кода
\end{itemize}
\end{block}
\begin{block}{Задачи}
\begin{itemize}
\item Изучить существующие подходы и алгоритмы специализации для низкоуровневых языков программирования
\item Разработать архитектуру специализатора с учётом рассмотреных подходов и особенностей языка специалиации
\item Добавление возможностей специализатора, необходимых для самоприменения 
\item Исследование возможностей полученного специазатора0
\end{itemize}
\end{block}
\end{frame}

\lstset{language=C}
\begin{frame}[fragile]\frametitle{Binding time analisys}
\color{red} динамичесий \color{ao} статический
\color{black}
\begin{columns}
\begin{column}{0.4\textwidth}
\begin{lstlisting}[
  belowcaptionskip=.2\baselineskip,
  breaklines=true,
  xleftmargin=\parindent,
  showstringspaces=false,
  basicstyle=\fontsize{8}{4}\selectfont\ttfamily,
  keywordstyle=\bfseries,
  commentstyle=\itshape\color{gray!40!black},
  stringstyle=\color{ao},
  numbers=left,
  numbersep=5pt,
  numberstyle=\fontsize{6}{4}\selectfont\ttfamily\color{gray},]
int pow(@\textcolor{red}{int a}@, @\textcolor{ao}{int b}@) {
    int res = 1;
    while (b > 0) {
        res = res * a;
        b = b - 1;
    }
    return a;
}
\end{lstlisting}
\end{column}
\begin{column}{0.2\textwidth}
$$\xrightarrow{\quad BTA \quad}$$
\end{column}
\begin{column}{0.4\textwidth}
\begin{lstlisting}[
  belowcaptionskip=.2\baselineskip,
  breaklines=true,
  xleftmargin=\parindent,
  showstringspaces=false,
  basicstyle=\fontsize{8}{4}\selectfont\ttfamily,
  keywordstyle=\bfseries,
  commentstyle=\itshape\color{gray!40!black},
  stringstyle=\color{ao},
  numbers=left,
  numbersep=5pt,
  numberstyle=\fontsize{6}{4}\selectfont\ttfamily\color{gray},]
int pow(@\textcolor{red}{int a}@, @\textcolor{ao}{int b}@) {
    @\textcolor{red}{int res = 1;}@
    while (@\textcolor{ao}{b > 0}@) {
        @\textcolor{red}{res = res * a}@;
        @\textcolor{ao}{b = b - 1}@;
    }
    @\textcolor{red}{return a}@;
}
\end{lstlisting}
\end{column}
\end{columns}
\end{frame}

\lstset{language=C}
\begin{frame}[fragile]\frametitle{Основные проблемы}
Проблема : BTA делает регистры динамическими \\
\begin{lstlisting}[
  belowcaptionskip=.2\baselineskip,
  breaklines=true,
  xleftmargin=\parindent,
  showstringspaces=false,
  basicstyle=\fontsize{9}{4}\selectfont\ttfamily,
  keywordstyle=\bfseries,
  commentstyle=\itshape\color{gray!40!black},
  stringstyle=\color{ao},
  numbers=left,
  numbersep=5pt,
  numberstyle=\fontsize{9}{4}\selectfont\ttfamily\color{gray},]
//%esi динамический
mov %esi %eax
//теперь %eax динамический
mov 4 %eax
//@специализатор   не   знает   значение@ %eax
\end{lstlisting}
Решение : Online специализация \\
\vfill
Проблема : Комплексные инструкции (push)\\\begin{lstlisting}[
  belowcaptionskip=.2\baselineskip,
  breaklines=true,
  xleftmargin=\parindent,
  showstringspaces=false,
  basicstyle=\fontsize{9}{4}\selectfont\ttfamily,
  keywordstyle=\bfseries,
  commentstyle=\itshape\color{gray!40!black},
  stringstyle=\color{ao},
  numbers=left,
  numbersep=5pt,
  numberstyle=\fontsize{9}{4}\selectfont\ttfamily\color{gray},]
//%esi динамический
push %esi
//теперь %esp динамический
push 4
pop %eax
//@специализатор   не   знает   значение@ %eax
\end{lstlisting}
Решение : BTA разделяет инструкцию на простые\\
\end{frame}


\lstset{language=C}
\begin{frame}[fragile]\frametitle{Основные проблемы}
Проблема : Статические адреса (например стек) становятся константами \\
\begin{lstlisting}[
  belowcaptionskip=.2\baselineskip,
  breaklines=true,
  xleftmargin=\parindent,
  showstringspaces=false,
  basicstyle=\fontsize{9}{4}\selectfont\ttfamily,
  keywordstyle=\bfseries,
  commentstyle=\itshape\color{gray!40!black},
  stringstyle=\color{ao},
  numbers=left,
  numbersep=5pt,
  numberstyle=\fontsize{9}{4}\selectfont\ttfamily\color{gray},]
//%esi динамический
push %esi
//@специализируется в@
mov %esi (268123094)
//@адрес может меняться между запусками@ %eax
\end{lstlisting}
\vfill
Решение : Символьные вычисления
\begin{lstlisting}[
  belowcaptionskip=.2\baselineskip,
  breaklines=true,
  xleftmargin=\parindent,
  showstringspaces=false,
  basicstyle=\fontsize{9}{4}\selectfont\ttfamily,
  keywordstyle=\bfseries,
  commentstyle=\itshape\color{gray!40!black},
  stringstyle=\color{ao},
  numbers=left,
  numbersep=5pt,
  numberstyle=\fontsize{9}{4}\selectfont\ttfamily\color{gray},]
//%esi динамический
push %esi
//@специализируется в@
mov %esi -48(0)
\end{lstlisting}
\end{frame}


\lstset{language=C}
\begin{frame}[fragile]\frametitle{Основные проблемы}
Проблема : Не всегда можно сделать lifting \\
\begin{lstlisting}[
  belowcaptionskip=.2\baselineskip,
  breaklines=true,
  xleftmargin=\parindent,
  showstringspaces=false,
  basicstyle=\fontsize{9}{4}\selectfont\ttfamily,
  keywordstyle=\bfseries,
  commentstyle=\itshape\color{gray!40!black},
  stringstyle=\color{ao},
  numbers=left,
  numbersep=5pt,
  numberstyle=\fontsize{9}{4}\selectfont\ttfamily\color{gray},]
lea -72(0) %ebx
add 10 %ebx
//%edx динамический
mov %ebx 4(%edx)
//@после специализации получим@
mov -62(0) 4(%ebx)
//@запись из памяти в память запрещена@
\end{lstlisting}
\vfill
Решение : Генерация дополнительных инструкций
\begin{lstlisting}[
  belowcaptionskip=.2\baselineskip,
  breaklines=true,
  xleftmargin=\parindent,
  showstringspaces=false,
  basicstyle=\fontsize{9}{4}\selectfont\ttfamily,
  keywordstyle=\bfseries,
  commentstyle=\itshape\color{gray!40!black},
  stringstyle=\color{ao},
  numbers=left,
  numbersep=5pt,
  numberstyle=\fontsize{9}{4}\selectfont\ttfamily\color{gray},]
lea -72(0) %ebx
add 10 %ebx
//%edx динамический
mov %ebx 4(%edx)
//@после специализации получим@
lea -62(0) %ebx // @сгенерированная инструкция@
mov %ebx 4(%ebx)
\end{lstlisting}
\end{frame}

\lstset{language=C}
\begin{frame}[fragile]\frametitle{KMP тест}
\begin{columns}
\begin{column}{0.2\textwidth}
\begin{lstlisting}[
  belowcaptionskip=.2\baselineskip,
  breaklines=true,
  xleftmargin=\parindent,
  showstringspaces=false,
  basicstyle=\fontsize{2}{2}\selectfont\ttfamily,
  keywordstyle=\bfseries,
  commentstyle=\itshape\color{gray!40!black},
  stringstyle=\color{ao},
  numbers=left,
  numbersep=5pt,
  numberstyle=\fontsize{2}{2}\selectfont\ttfamily\color{gray},]
int kmp(char* p, char* d, char* free1, char* free2) {
    char* pp = p;
    char* f = free1;
    char* ff = free1;
    char* neg = free2;
    char* f0 = free1;
    char* neg0 = free2;
    while (1) {
        if (p[0] == 0) {
            return 1;
        }
        else if (f == f0) {
            if (member3(p[0], neg, neg0)) {
                if (ff == f0) {
                    p = pp;
                    d++;
                    ff = f0;
                    neg = neg0;
                    continue;
                }
                else {
                    p = pp;
                    ff++;
                    f = ff;
                    continue;
                }
            }
            else if (neg == neg0 && d[0] == 0) {
                return 0;
            }
            else if (p[0] == d[0]) {
                char* ptr = ff;
                while (ptr != f0) {
                    ptr[-1] = ptr[0];
                    ptr++;
                }
                ptr[-1] = p[0];
                ff--;
                p++;
                d++;
                neg = neg0;
                continue;
            }
            else if (ff == f0) {
                p = pp;
                d++;
                f = f0;
                ff = f0;
                neg = neg0;
                continue;
            }
            else {
                neg--;
                neg[0] = p[0];
                p = pp;
                ff++;
                f = ff;
                continue;
            }
        }
        else if (p[0] == f[0]) {
            p++;
            f++;
            continue;
        }
        else {
            p = pp;
            ff++;
            f = ff;
            continue;
        }
    }
}
\end{lstlisting}
\end{column}
\begin{column}{0.4\textwidth}
$$\xrightarrow{\quad \llbracket spec \rrbracket_{ASM}[kmp, "a"] \quad}$$
\end{column}
\begin{column}{0.4\textwidth}
\begin{lstlisting}[
  belowcaptionskip=.2\baselineskip,
  breaklines=true,
  xleftmargin=\parindent,
  showstringspaces=false,
  basicstyle=\fontsize{9}{4}\selectfont\ttfamily,
  keywordstyle=\bfseries,
  commentstyle=\itshape\color{gray!40!black},
  stringstyle=\color{ao},
  numbers=left,
  numbersep=5pt,
  numberstyle=\fontsize{6}{4}\selectfont\ttfamily\color{gray},]
Start block -633763
mov89 %rsi  -88(0) 
mov8b -88(0)  %rax 
movb6 0(rax)  %rax 
test %al %al
cjump 0x85 to 565830
premov 0 , %rax 
ret

Start block 565830
mov8b -88(0)  %rax 
movb6 0(rax)  %rax 
cmp39 97  %rax 
cjump 0x85 to 925494
add83 -88(0)  1
premov 1 , %rax 
ret

Start block 925494
add83 -88(0)  1
mov8b -88(0)  %rax 
movb6 0(rax)  %rax 
test %al %al
cjump 0x85 to 565830
premov 0 , %rax 
ret
\end{lstlisting}
\end{column}
\end{columns}
\end{frame}

\lstset{language=C}
\begin{frame}[fragile]\frametitle{Специализация интерпретатора}
\begin{columns}
\begin{column}{0.4\textwidth}
\begin{lstlisting}[
  belowcaptionskip=.2\baselineskip,
  breaklines=true,
  xleftmargin=\parindent,
  showstringspaces=false,
  basicstyle=\fontsize{8}{4}\selectfont\ttfamily,
  keywordstyle=\bfseries,
  commentstyle=\itshape\color{gray!40!black},
  stringstyle=\color{ao},
  numbers=left,
  numbersep=5pt,
  numberstyle=\fontsize{8}{4}\selectfont\ttfamily\color{gray},]
int foo(@\textcolor{blue}{int a, int b}@){
    int c;
    @\textcolor{red}{c = a}@;
    @\textcolor{ao}{c += b}@;
    return c;
}
\end{lstlisting}
\end{column}
\begin{column}{0.4\textwidth}
$$\xrightarrow{ \llbracket spec \rrbracket_{ASM}[interpreter, foo] }$$
\end{column}
\begin{column}{0.3\textwidth}
\begin{lstlisting}[
  belowcaptionskip=.2\baselineskip,
  breaklines=true,
  xleftmargin=\parindent,
  showstringspaces=false,
  basicstyle=\fontsize{6}{4}\selectfont\ttfamily,
  keywordstyle=\bfseries,
  commentstyle=\itshape\color{gray!40!black},
  stringstyle=\color{ao},
  numbers=left,
  numbersep=5pt,
  numberstyle=\fontsize{3}{4}\selectfont\ttfamily\color{gray},]
Start block -697046
mov89 %rsi  -72(0) 
call malloc
@\textcolor{blue}{mov8b -72(0)  \%rax}@
@\textcolor{blue}{mov8b 0(rax)  \%rax}@
@\textcolor{blue}{mov89 \%rax  0(2)}@
@\textcolor{blue}{add83 -72(0)  4}@
@\textcolor{blue}{mov8b -72(0)  \%rax}@
@\textcolor{blue}{mov8b 0(rax)  \%rax}@
@\textcolor{blue}{mov89 \%rax  16(2)}@
@\textcolor{blue}{add83 -72(0)  4}@
@\textcolor{red}{mov8b 0(2)  \%rax}@ 
@\textcolor{red}{mov89 \%rax  -100(0)}@ 
@\textcolor{red}{mov8b -100(0)  \%rax}@ 
@\textcolor{red}{mov89 \%rax  -28(0)}@ 
@\textcolor{red}{mov8b -28(0)  \%rax}@ 
@\textcolor{red}{mov89 \%rax  32(2)}@
@\textcolor{ao}{mov8b 16(2)  \%rax}@ 
@\textcolor{ao}{mov89 \%rax  -100(0)}@ 
@\textcolor{ao}{mov8b -100(0)  \%rax}@ 
@\textcolor{ao}{mov89 \%rax  -28(0)}@ 
@\textcolor{ao}{mov8b 32(2)  \%rcx}@ 
@\textcolor{ao}{mov8b -28(0)  \%rdx}@ 
@\textcolor{ao}{add01 \%rdx  \%rcx}@ 
@\textcolor{ao}{mov89 \%rdx  32(2)}@ 
mov8b 32(2)  %rax 
mov89 %rax  -100(0) 
mov8b -100(0)  %rax 
ret

\end{lstlisting}
\end{column}
\end{columns}
\end{frame}

\begin{frame}\frametitle{Итоги}
\begin{itemize}
\item Создана модель специализатора
\vfill
\item Произведено тестирование контрольных точек и решены ключевые проблемы
\vfill
\item Протестирована первая проекция футамуры
\vfill
\item Вторая проекция футамуры не завершена
\end{itemize}
\end{frame}

\begin{frame}\frametitle{Конец}
https://github.com/XJIE6/spec
\end{frame}
\end{document}