%% Простая презентация с примером включения программного кода и
%% пошаговых спецэффектов
\documentclass[demo]{beamer}
\usepackage{amsmath}
\usepackage{fontspec}
\usepackage{xunicode}
\usepackage{xltxtra}
\usepackage{xecyr}
\usepackage{hyperref}
\usepackage{ stmaryrd }
\usepackage{tikz}
\setmainfont[Mapping=tex-text]{DejaVu Serif}
\setsansfont[Mapping=tex-text]{DejaVu Sans}
\setmonofont[Mapping=tex-text]{DejaVu Sans Mono}
\usepackage{polyglossia}
\setdefaultlanguage{russian}
\usepackage{graphicx}
\addtobeamertemplate{navigation symbols}{}{%
    \usebeamerfont{footline}%
    \usebeamercolor[fg]{footline}%
    \hspace{1em}%
    \insertframenumber/\inserttotalframenumber
}
\usepackage{listings}
\lstdefinestyle{mycode}{
  belowcaptionskip=1\baselineskip,
  breaklines=true,
  xleftmargin=\parindent,
  showstringspaces=false,
  basicstyle=\footnotesize\ttfamily,
  keywordstyle=\bfseries,
  commentstyle=\itshape\color{gray!40!black},
  stringstyle=\color{red},
  numbers=left,
  numbersep=5pt,
  numberstyle=\tiny\color{gray},
}
\lstset{escapechar=@,style=mycode}
\newcommand{\xdownarrow}[1]{%
  {\left\downarrow\vbox to #1{}\right.\kern-\nulldelimiterspace}
}

\begin{document}
\title{Специализация машинного кода}
\author{Юрий Кравченко\\{\footnotesize\textcolor{gray}{руководитель Березун Даниил Андреевич}}}
\institute{СПбАУ}
\frame{\titlepage}

\begin{frame}\frametitle{Специализация}
  \begin{block}{Традиционное исполнение программы}
    $$\llbracket p \rrbracket_L [in_1, in_2, \dots] = out$$
  \end{block}
  \begin{block}{Специализатор}
    Программу $spec$ назовём специализатором, если
    $$\begin{array}{l l c l}
        \llbracket spec \rrbracket_{L_2} & [p, in_1] &= &p_{spec}\\
        \llbracket p_{spec} \rrbracket_{L_1} & [in_2,\dots] &= &out
      \end{array}$$
    \end{block}
\end{frame}

\begin{frame}{Цель специализации}
  $source$ --- программа на языке S\\
  $int$ --- интерпретатор для языка S на языке L\\
  \vfill
  \begin{block}{Проекции Футамуры \hfill [1983]}
    $$
    \begin{array}{l l l c l}
      I & \llbracket spec \rrbracket_L &[int, source] &= &target\\
      II & \llbracket spec \rrbracket_L &[spec, int] &= &comp\\
      III & \llbracket spec \rrbracket_L &[spec, spec] &= &cogen
    \end{array}$$
  \end{block}
  \vfill
  \begin{block}{Вывод}
    $$Interpreter \overset{spec}{\longrightarrow} Compiler$$
  \end{block}
\end{frame}

\begin{frame}\frametitle{В чём подвох?}
  \begin{itemize}
  \item{ Компилятор в язык реализации интерпретатора
      $$Interpreter^{S}_{\mathbf{L}}
      \xrightarrow{\qquad spec^{\mathbf{L}}_{\mathbf{L}}\qquad} Compiler^{S \to \mathbf{L}}_{\mathbf{L}}$$
      \hfill {\large \color{red}Это основная проблема}
    }
    \vfill
  \item {Апостериорный факт: реализовывать специализаторы сложно ($\sim$ как компиляторы)}
  \end{itemize}
\end{frame}

\begin{frame}{Релевантные исследования}
  Все текущие исследования имеют одну из двух проблем
  \begin{itemize}
  \item Искусственный язык
    \vfill
  \item Нет возможности самоприменения
  \end{itemize}
  \vfill
  Partial Evaluation of Machine Code \hfill [2015]
    \begin{itemize}
    \item Подмножество IA-32
    \vfill
    \item Использование сторонней закрытой библиотеки
    \vfill
    \item \large Написан на Java $\Rightarrow$ нельзя самоприменить
  	\end{itemize}
\end{frame}

\begin{frame}\frametitle{Идея}
  \begin{itemize}
  \item {Специализатор для машинного кода
      $$Interpreter^{S}_{\mathbf{ASM}}
      \xrightarrow{\qquad spec^{ \mathbf{ASM}}_{\mathbf{ASM}}\qquad} Compiler^{S \to \mathbf{ASM}}$$
    }
    \vfill
  \item {Как получить $Interpreter^{S}_{\mathbf{ASM}}$?
      $$\llbracket gcc \rrbracket [Interpreter^{S}_{\textsc{C}}] =
      Interpreter^{S}_{\mathbf{ASM}}$$}
    \vfill
  \item {Как получить $spec^{\mathbf{ASM}}_{\mathbf{ASM}}$?
      $$\llbracket gcc \rrbracket [spec^{ \mathbf{ASM}}_{\textsc{C}}] =
      spec^{\mathbf{ASM}}_{\mathbf{ASM}}$$}
    \vfill
  \item{Как получить $spec^{\mathbf{ASM}}_{\textsc{C}}$?
      \only{\hspace{1cm}---\hspace{1cm} \textbf{\Large\color{green}Цель}}}
    \vfill
  \end{itemize}
\end{frame}

\begin{frame}\frametitle{Детали реализации}
\begin{itemize}
\item Статические инструкции вычисляются
\vfill
\item Динамические инструкции упрощаются и записываются в ответ
\vfill
\item При динамических условиях вычисление раздваивается
\vfill
\item Если состояние уже было обработано ранее, специализация останавливается
\end{itemize}

\end{frame}

\begin{frame}\frametitle{Начальный этап}
\begin{itemize}
    \item {Тренировочный специализатор для flowchart}
    \vfill
    \item {Изучение AMD64 Architecture Programmer’s Manual Volume 3: General-Purpose and System Instructions}
    \vfill
    \item {Выбрано подмножество инструкций: add, cmp, test, imul, jmp, jcc, mov, lea, pop, leave, push, ret, sub, call, ...}
    \vfill
    \item {Разработка архитектуры проекта}
\end{itemize}
\end{frame}

\lstset{language=C}
\begin{frame}[fragile]\frametitle{Интерпретация}
\begin{lstlisting}
void sort(int len, int* a) {
    for (int i = 0; i < len; ++i) {
        for (int j = i + 1; j < len; ++j) {
            if (a[i] > a[j]) {
                int k = a[i];
                a[i] = a[j];
                a[j] = k;
            }
        }
    }
}
\end{lstlisting}
\end{frame}

\lstset{language=C}
\begin{frame}[fragile]\frametitle{Работа с памятью}
\begin{lstlisting}
char* eratosphen(int n) {
    char* a = my_malloc(n);
    for (int i = 0; i < n; ++i) {
        a[i] = 1;
    }
    a[0] = 0;
    a[1] = 0;
    for (int i = 2; i < n; ++i) {
        if (a[i]) {
            for (int j = i * i; j < n; j += i) {
                a[j] = 0;
            }
        }
    }
    return a;
}
\end{lstlisting}
\end{frame}

\lstset{language=C}
\begin{frame}[fragile]\frametitle{Простая специализация}
\begin{lstlisting}
int dict(int len, int* keys, int* values, int key) {
    for (int i = 0; i < len; ++i) {
        if (keys[i] == key) {
            return value[i];
        }
    }
    return -1;
}
\end{lstlisting}
\vfill
$\Bigg\downarrow{\qquad \llbracket spec \rrbracket_{ASM}[dict, [3, [0, 1, 2], ?, 1]] \qquad}$
\vfill
\begin{lstlisting}
Start block 701084
mov89 %rdx  -48(0) 
mov8b -48(0)  %rax 
add01 4  %rax 
mov8b 0(rax)  %rax 
ret
\end{lstlisting}

\end{frame}

\lstset{language=C}
\begin{frame}[fragile]\frametitle{Полный цикл обработки}
\begin{lstlisting}
int pow(int a, int b) {
    if (b == 0) {
        return 1;
    }
    return my_pow(a, b - 1) * a;
}
\end{lstlisting}
\vfill
$\Bigg\downarrow{\qquad \llbracket spec \rrbracket_{ASM}[pow, 5] \qquad}$
\vfill
\tiny 89 7D FC 8B 45 FC 89 C7 89 7D FC 8B 45 FC 89 C7 89 7D FC 8B 45 FC 89 C7 89 7D FC 8B 45 FC 89 C7 89 7D FC 8B 45 FC 89 C7 89 7D FC B8 01 00 00 00 0F AF 45 FC 0F AF 45 FC 0F AF 45 FC 0F AF 45 FC 0F AF 45 FC C3
\end{frame}

\lstset{language=C}
\begin{frame}[fragile]\frametitle{KMP тест}
\begin{columns}
\begin{column}{0.2\textwidth}
\begin{lstlisting}[
  belowcaptionskip=.2\baselineskip,
  breaklines=true,
  xleftmargin=\parindent,
  showstringspaces=false,
  basicstyle=\fontsize{2}{2}\selectfont\ttfamily,
  keywordstyle=\bfseries,
  commentstyle=\itshape\color{gray!40!black},
  stringstyle=\color{red},
  numbers=left,
  numbersep=5pt,
  numberstyle=\fontsize{2}{2}\selectfont\ttfamily\color{gray},]
int kmp(char* p, char* d, char* free1, char* free2) {
    char* pp = p;
    char* f = free1;
    char* ff = free1;
    char* neg = free2;
    char* f0 = free1;
    char* neg0 = free2;
    while (1) {
        if (p[0] == 0) {
            return 1;
        }
        else if (f == f0) {
            if (member3(p[0], neg, neg0)) {
                if (ff == f0) {
                    p = pp;
                    d++;
                    ff = f0;
                    neg = neg0;
                    continue;
                }
                else {
                    p = pp;
                    ff++;
                    f = ff;
                    continue;
                }
            }
            else if (neg == neg0 && d[0] == 0) {
                return 0;
            }
            else if (p[0] == d[0]) {
                char* ptr = ff;
                while (ptr != f0) {
                    ptr[-1] = ptr[0];
                    ptr++;
                }
                ptr[-1] = p[0];
                ff--;
                p++;
                d++;
                neg = neg0;
                continue;
            }
            else if (ff == f0) {
                p = pp;
                d++;
                f = f0;
                ff = f0;
                neg = neg0;
                continue;
            }
            else {
                neg--;
                neg[0] = p[0];
                p = pp;
                ff++;
                f = ff;
                continue;
            }
        }
        else if (p[0] == f[0]) {
            p++;
            f++;
            continue;
        }
        else {
            p = pp;
            ff++;
            f = ff;
            continue;
        }
    }
}
\end{lstlisting}
\end{column}
\begin{column}{0.4\textwidth}
$$\xrightarrow{\quad \llbracket spec \rrbracket_{ASM}[kmp, "a"] \quad}$$
\end{column}
\begin{column}{0.4\textwidth}
\begin{lstlisting}[
  belowcaptionskip=.2\baselineskip,
  breaklines=true,
  xleftmargin=\parindent,
  showstringspaces=false,
  basicstyle=\fontsize{9}{4}\selectfont\ttfamily,
  keywordstyle=\bfseries,
  commentstyle=\itshape\color{gray!40!black},
  stringstyle=\color{red},
  numbers=left,
  numbersep=5pt,
  numberstyle=\fontsize{6}{4}\selectfont\ttfamily\color{gray},]
Start block -633763
mov89 %rsi  -88(0) 
mov8b -88(0)  %rax 
movb6 0(rax)  %rax 
test %al %al
cjump 0x85 to 565830
premov 0 , %rax 
ret

Start block 565830
mov8b -88(0)  %rax 
movb6 0(rax)  %rax 
cmp39 97  %rax 
cjump 0x85 to 925494
add83 -88(0)  1
premov 1 , %rax 
ret

Start block 925494
add83 -88(0)  1
mov8b -88(0)  %rax 
movb6 0(rax)  %rax 
test %al %al
cjump 0x85 to 565830
premov 0 , %rax 
ret
\end{lstlisting}
\end{column}
\end{columns}
\end{frame}

\begin{frame}\frametitle{Планы на будущее}

\begin{itemize}
\item Специализировать интерпретатор
\vfill 
\item Специализировать специализатор
\vfill
\item Сравнить производительность
\end{itemize}

\end{frame}

\begin{frame}\frametitle{Конец}
https://github.com/XJIE6/spec
\end{frame}
\end{document}
%%% Local Variables:
%%% mode: latex
%%% TeX-master: t
%%% End:
