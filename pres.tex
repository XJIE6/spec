\documentclass{beamer}
\usecolortheme{crane}
\usepackage{amsmath}
\usepackage{fontspec}
\usepackage{xunicode}
\usepackage{xltxtra}
\usepackage{xecyr}
\usepackage{hyperref}
\usepackage{multirow}
\usepackage{ stmaryrd }
\usepackage{array}
\usepackage{booktabs}
\usepackage{tikz}
\setmainfont[Mapping=tex-text]{DejaVu Serif}
\setsansfont[Mapping=tex-text]{DejaVu Sans}
\setmonofont[Mapping=tex-text]{DejaVu Sans Mono}
\usepackage{polyglossia}
\setdefaultlanguage{russian}
\usepackage{graphicx}
\definecolor{ao}{rgb}{0.0, 0.4, 0.0}
\definecolor{br}{rgb}{0.4, 0.2, 0.0}
\addtobeamertemplate{navigation symbols}{}{%
    \usebeamerfont{fotline}%
    \usebeamercolor[fg]{footline}%
    \hspace{1em}%
    \insertframenumber/\inserttotalframenumber
}
\usepackage{listings}
\lstdefinestyle{mycode}{
  belowcaptionskip=1\baselineskip,
  breaklines=true,
  xleftmargin=\parindent,
  showstringspaces=false,
  basicstyle=\footnotesize\ttfamily,
  keywordstyle=\bfseries,
  commentstyle=\itshape\color{gray!40!black},
  stringstyle=\color{ao},
  numbers=left,
  numbersep=5pt,
  numberstyle=\tiny\color{gray},
}
\lstset{escapechar=@,style=mycode}
\newcommand{\xdownarrow}[1]{%
  {\left\downarrow\vbox to #1{}\right.\kern-\nulldelimiterspace}
}

\begin{document}
\title{Специализация машинного кода}
\author{Юрий Кравченко\\{\footnotesize\textcolor{gray}{руководитель Березун Даниил Андреевич}}}
\institute{Лаборатория языковых инструментов}
\frame{\titlepage}

\begin{frame}\frametitle{Специализация}
  \begin{block}{Традиционное исполнение программы}
    $$\llbracket p \rrbracket_L [in_1, in_2, \dots] = out$$
  \end{block}
  \begin{block}{Специализатор}
    Программу $spec$ назовём специализатором, если
    $$\begin{array}{l l c l}
        \llbracket spec \rrbracket_{L_2} & [p, \color{ao}in_1\color{black}] &= &p_{spec}\\
        \llbracket p_{spec} \rrbracket_{L_1} & [\color{red}in_2 \color{black},\dots] &= &out
      \end{array}$$
      {\hfill \color{red} динамический \hfill \color{ao} статический \hfill}
    \end{block}
\end{frame}

\begin{frame}{Цель специализации}
  $source$ --- программа на языке S\\
  $int$ --- интерпретатор для языка S на языке L\\
  \vfill
  \begin{block}{Проекции Футамуры \hfill [1973]}
    $$
    \begin{array}{l l l c l}
      I & \llbracket spec \rrbracket_L &[int, source] &= &target\\
      II & \llbracket spec \rrbracket_L &[spec, int] &= &comp\\
      III & \llbracket spec \rrbracket_L &[spec, spec] &= &cogen
    \end{array}$$
  \end{block}
  \vfill
  \begin{block}{Вывод}
    $$Interpreter \overset{spec}{\longrightarrow} Compiler$$
  \end{block}
\end{frame}

\begin{frame}\frametitle{Почему не используется}
  \begin{itemize}
  \item{ Компилятор в язык реализации интерпретатора
      $$Interpreter^{S}_{\mathbf{L}}
      \xrightarrow{\qquad spec^{\mathbf{L}}_{\mathbf{L}}\qquad} Compiler^{S \to \mathbf{L}}_{\mathbf{L}}$$
      \hfill {\large \color{red}Это основная проблема}
    }
    \vfill
  \item {Апостериорный факт: реализовывать специализаторы сложно ($\sim$ как компиляторы)}
  \end{itemize}
\end{frame}

\begin{frame}\frametitle{Идея}
  \begin{itemize}
  \item {Специализатор для машинного кода
      $$Interpreter^{S}_{\mathbf{ASM}}
      \xrightarrow{\qquad spec^{ \mathbf{ASM}}_{\mathbf{ASM}}\qquad} Compiler^{S \to \mathbf{ASM}}$$
    }
    \vfill
  \item {Как получить $Interpreter^{S}_{\mathbf{ASM}}$?
      $$\llbracket gcc \rrbracket [Interpreter^{S}_{\textsc{C}}] =
      Interpreter^{S}_{\mathbf{ASM}}$$}
    \vfill
  \item {Как получить $spec^{\mathbf{ASM}}_{\mathbf{ASM}}$?
      $$\llbracket gcc \rrbracket [spec^{ \mathbf{ASM}}_{\textsc{C}}] =
      spec^{\mathbf{ASM}}_{\mathbf{ASM}}$$}
    \vfill
  \item{Как получить $spec^{\mathbf{ASM}}_{\textsc{C}}$? \pause \hfill {\large \color{red} --- \hfill Цель} \hfill}
    \vfill
  \end{itemize}
\end{frame}

\begin{frame}{Существующие подходы}
	Partial evaluation and automatic program generation, 
	Neil D. Jones, Carsten K. Gomard, Peter Sestoft, 1994. 
    \begin{itemize}
    \vfill
    \item Специализатор для языка Flow Chart
    \vfill
    \item Flow Chart структурно похож на машинный код
    \vfill
    \item Специализатор является самоприменимым
  	\end{itemize}
  \vfill
    Partial evaluation of machine code,
    Srinivasan Venkatesh, Reps Thomas, 2015.
    \vfill
    \begin{itemize}
    \item Специализатор для подмножества IA-32
    \vfill
    \item Использованы сложные техники
    \vfill
    \item Реализован на Java $\Rightarrow$ нельзя самоприменить
  	\end{itemize}
\end{frame}

\begin{frame}[fragile]\frametitle{Дипломная работа}
\begin{itemize}
\item Исследованы различные подходы специализации низкоуровневых языков
\vfill
\item Исследованы особенности специализации машинного кода
\vfill
\item Предложен алгоритм специализации машинного кода
\vfill
\item Реализован прототип специализатора на языке С и произведена его апробация
\vfill
\item Аппробирована первая проекция проекция футамуры для маленького языка
\end{itemize}
\end{frame}

\lstset{language=C}
\begin{frame}[fragile]\frametitle{Особенности специализации машинного кода}
Проблема : BTA делает регистры динамическими \\
\begin{lstlisting}[
  belowcaptionskip=.2\baselineskip,
  breaklines=true,
  xleftmargin=\parindent,
  showstringspaces=false,
  basicstyle=\fontsize{9}{4}\selectfont\ttfamily,
  keywordstyle=\bfseries,
  commentstyle=\itshape\color{gray!40!black},
  stringstyle=\color{ao},
  numbers=left,
  numbersep=5pt,
  numberstyle=\fontsize{9}{4}\selectfont\ttfamily\color{gray},]
//%esi динамический
mov %esi %eax
//теперь %eax динамический
mov 4 %eax
//@специализатор   не   знает   значение@ %eax
\end{lstlisting}
Решение : Online специализация \\
\vfill
Проблема : Комплексные инструкции (push)\\
\begin{lstlisting}[
  belowcaptionskip=.2\baselineskip,
  breaklines=true,
  xleftmargin=\parindent,
  showstringspaces=false,
  basicstyle=\fontsize{9}{4}\selectfont\ttfamily,
  keywordstyle=\bfseries,
  commentstyle=\itshape\color{gray!40!black},
  stringstyle=\color{ao},
  numbers=left,
  numbersep=5pt,
  numberstyle=\fontsize{9}{4}\selectfont\ttfamily\color{gray},]
//%esi динамический
push %esi
//теперь %esp динамический
push 4
pop %eax
//@специализатор   не   знает   значение@ %eax
\end{lstlisting}
Решение : BTA разделяет инструкцию на простые\\
\end{frame}


\lstset{language=C}
\begin{frame}[fragile]\frametitle{Особенности машинного кода}
Проблема : Специализация констант времени исполнения \\
\begin{lstlisting}[
  belowcaptionskip=.2\baselineskip,
  breaklines=true,
  xleftmargin=\parindent,
  showstringspaces=false,
  basicstyle=\fontsize{9}{4}\selectfont\ttfamily,
  keywordstyle=\bfseries,
  commentstyle=\itshape\color{gray!40!black},
  stringstyle=\color{ao},
  numbers=left,
  numbersep=5pt,
  numberstyle=\fontsize{9}{4}\selectfont\ttfamily\color{gray},]
//%esi динамический
push %esi
//@специализируется в@
mov %esi (268123094)
//@адрес может меняться между запусками@ %eax
\end{lstlisting}
\vfill
Решение : Символьные вычисления
\begin{lstlisting}[
  belowcaptionskip=.2\baselineskip,
  breaklines=true,
  xleftmargin=\parindent,
  showstringspaces=false,
  basicstyle=\fontsize{9}{4}\selectfont\ttfamily,
  keywordstyle=\bfseries,
  commentstyle=\itshape\color{gray!40!black},
  stringstyle=\color{ao},
  numbers=left,
  numbersep=5pt,
  numberstyle=\fontsize{9}{4}\selectfont\ttfamily\color{gray},]
//%esi динамический
push %esi
//@специализируется в@
mov %esi -48(0)
\end{lstlisting}
\end{frame}

\begin{frame}[fragile]\frametitle{А что сейчас?}
\begin{columns}
\begin{column}{0.5\textwidth}
\begin{itemize}
\item есть функции
\vfill
\item for, while, if
\vfill
\item нет структур данных
\vfill
\item один тип - int
\end{itemize}
\end{column}
\begin{column}{0.5\textwidth}
\begin{lstlisting}[
  belowcaptionskip=.2\baselineskip,
  breaklines=true,
  xleftmargin=\parindent,
  showstringspaces=false,
  basicstyle=\fontsize{8}{4}\selectfont\ttfamily,
  keywordstyle=\bfseries,
  commentstyle=\itshape\color{gray!40!black},
  stringstyle=\color{ao},
  numbers=left,
  numbersep=5pt,
  numberstyle=\fontsize{6}{4}\selectfont\ttfamily\color{gray},]
fun foo (a, b)
begin
    if (b == 0)
    then
        return 1
    else
        c := foo(a, b - 1);
        return a * c
    fi
end

a := 7;
b := 5;
return foo(a, b) 
\end{lstlisting}
\end{column}
\end{columns}
\end{frame}


\begin{frame}[fragile]\frametitle{Межпроцедурность}

\vfill
\begin{columns}
\begin{column}{0.5\textwidth}
\begin{lstlisting}[
  belowcaptionskip=.2\baselineskip,
  breaklines=true,
  xleftmargin=\parindent,
  showstringspaces=false,
  basicstyle=\fontsize{8}{4}\selectfont\ttfamily,
  keywordstyle=\bfseries,
  commentstyle=\itshape\color{gray!40!black},
  stringstyle=\color{ao},
  numbers=left,
  numbersep=5pt,
  numberstyle=\fontsize{6}{4}\selectfont\ttfamily\color{gray},]
@\textcolor{red}{некое состояние программы}@
@\textcolor{red}{rdx - динамический}@
mov rdx rcx 
call foo
@\textcolor{red}{какое состояние?}@
\end{lstlisting}
\end{column}
\begin{column}{0.5\textwidth}
\begin{lstlisting}[
  belowcaptionskip=.2\baselineskip,
  breaklines=true,
  xleftmargin=\parindent,
  showstringspaces=false,
  basicstyle=\fontsize{8}{4}\selectfont\ttfamily,
  keywordstyle=\bfseries,
  commentstyle=\itshape\color{gray!40!black},
  stringstyle=\color{ao},
  numbers=left,
  numbersep=5pt,
  numberstyle=\fontsize{6}{4}\selectfont\ttfamily\color{gray},]
int foo(int @\textcolor{red}{a}@) {
    if (@\textcolor{red}{a}@ > 0) {
        return 1;
    }
    return 0;
}
\end{lstlisting}
\end{column}
\end{columns}
\vfill
\begin{columns}
\begin{column}{\textwidth}
\begin{lstlisting}[
  belowcaptionskip=.2\baselineskip,
  breaklines=true,
  xleftmargin=\parindent,
  showstringspaces=false,
  basicstyle=\fontsize{8}{4}\selectfont\ttfamily,
  keywordstyle=\bfseries,
  commentstyle=\itshape\color{gray!40!black},
  stringstyle=\color{ao},
  numbers=left,
  numbersep=5pt,
  numberstyle=\fontsize{6}{4}\selectfont\ttfamily\color{gray},]
int bar(int @\textcolor{ao}{a}@, int @\textcolor{red}{b}@) {
    if (@\textcolor{red}{b}@ == 0) {
        return 1;
    }
    return bar(@\textcolor{ao}{a}@, @\textcolor{red}{b - 1}@) * @\textcolor{ao}{a}@;
}
\end{lstlisting}
\end{column}
\end{columns}
\vfill
\end{frame}


\begin{frame}[fragile]\frametitle{Первая проекция}
\footnotesize{$$\llbracket spec \rrbracket_L [int, source] = target$$}
\begin{columns}
\begin{column}{0.5\textwidth}
\begin{lstlisting}[
  belowcaptionskip=.2\baselineskip,
  breaklines=true,
  xleftmargin=\parindent,
  showstringspaces=false,
  basicstyle=\fontsize{8}{4}\selectfont\ttfamily,
  keywordstyle=\bfseries,
  commentstyle=\itshape\color{gray!40!black},
  stringstyle=\color{ao},
  numbers=left,
  numbersep=5pt,
  numberstyle=\fontsize{6}{4}\selectfont\ttfamily\color{gray},]
return x + 2
\end{lstlisting}
\end{column}
\begin{column}{0.5\textwidth}
\begin{lstlisting}[
  belowcaptionskip=.2\baselineskip,
  breaklines=true,
  xleftmargin=\parindent,
  showstringspaces=false,
  basicstyle=\fontsize{7}{3}\selectfont\ttfamily,
  keywordstyle=\bfseries,
  commentstyle=\itshape\color{gray!40!black},
  stringstyle=\color{ao},
  numbers=left,
  numbersep=5pt,
  numberstyle=\fontsize{6}{4}\selectfont\ttfamily\color{gray},]
Start block 1
call 2
mov8b 12(2)  %rax 
ret

Start block 2
call 3
mov89 %rax  %rdx 
mov89 %rdx  12(2) 
premov 0(2) %rax 
ret

Start block 3
call 4
mov89 %rax  -80(0)
call 5
mov8b -80(0)  %rax 
add01 2  %rax
ret

Start block 4
mov8b 108(2)  %rax 
ret

Start block 5
premov 2 %rax 
ret
\end{lstlisting}
\end{column}
\end{columns}
\vfill
\end{frame}


\begin{frame}[fragile]\frametitle{Планы на будущее}
\vfill
Вторая проекция $$\llbracket spec \rrbracket_L [spec, int] = comp$$
\begin{itemize}
\vfill
\item Модифицировать специализатор
\vfill
\item Раздвоить специализатор
\vfill
\item Сравнить с компилятором и первой проекцией
\vfill
\end{itemize}
\end{frame}

\end{document}