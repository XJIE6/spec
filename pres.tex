\documentclass{beamer}
\usepackage{amsmath}
\usepackage{fontspec}
\usepackage{xunicode}
\usepackage{xltxtra}
\usepackage{xecyr}
\usepackage{hyperref}
\usepackage{multirow}
\usepackage{ stmaryrd }
\usepackage{array}
\usepackage{booktabs}
\usepackage{tikz}
\setmainfont[Mapping=tex-text]{DejaVu Serif}
\setsansfont[Mapping=tex-text]{DejaVu Sans}
\setmonofont[Mapping=tex-text]{DejaVu Sans Mono}
\usepackage{polyglossia}
\setdefaultlanguage{russian}
\usepackage{graphicx}
\definecolor{ao}{rgb}{0.0, 0.4, 0.0}
\definecolor{br}{rgb}{0.4, 0.2, 0.0}
\addtobeamertemplate{navigation symbols}{}{%
    \usebeamerfont{footline}%
    \usebeamercolor[fg]{footline}%
    \hspace{1em}%
    \insertframenumber/\inserttotalframenumber
}
\usepackage{listings}
\lstdefinestyle{mycode}{
  belowcaptionskip=1\baselineskip,
  breaklines=true,
  xleftmargin=\parindent,
  showstringspaces=false,
  basicstyle=\footnotesize\ttfamily,
  keywordstyle=\bfseries,
  commentstyle=\itshape\color{gray!40!black},
  stringstyle=\color{ao},
  numbers=left,
  numbersep=5pt,
  numberstyle=\tiny\color{gray},
}
\lstset{escapechar=@,style=mycode}
\newcommand{\xdownarrow}[1]{%
  {\left\downarrow\vbox to #1{}\right.\kern-\nulldelimiterspace}
}

\begin{document}
\title{Специализация машинного кода}
\author{Юрий Кравченко\\{\footnotesize\textcolor{gray}{руководитель Березун Даниил Андреевич}}}
\institute{СПбАУ}
\frame{\titlepage}

\begin{frame}\frametitle{Специализация}
  \begin{block}{Традиционное исполнение программы}
    $$\llbracket p \rrbracket_L [in_1, in_2, \dots] = out$$
  \end{block}
  \begin{block}{Специализатор}
    Программу $spec$ назовём специализатором, если
    $$\begin{array}{l l c l}
        \llbracket spec \rrbracket_{L_2} & [p, \color{ao}in_1\color{black}] &= &p_{spec}\\
        \llbracket p_{spec} \rrbracket_{L_1} & [\color{red}in_2 \color{black},\dots] &= &out
      \end{array}$$
      {\hfill \color{red} динамический \hfill \color{ao} статический \hfill}
    \end{block}
\end{frame}

\begin{frame}{Цель специализации}
  $source$ --- программа на языке S\\
  $int$ --- интерпретатор для языка S на языке L\\
  \vfill
  \begin{block}{Проекции Футамуры \hfill [1973]}
    $$
    \begin{array}{l l l c l}
      I & \llbracket spec \rrbracket_L &[int, source] &= &target\\
      II & \llbracket spec \rrbracket_L &[spec, int] &= &comp\\
      III & \llbracket spec \rrbracket_L &[spec, spec] &= &cogen
    \end{array}$$
  \end{block}
  \vfill
  \begin{block}{Вывод}
    $$Interpreter \overset{spec}{\longrightarrow} Compiler$$
  \end{block}
\end{frame}

\begin{frame}\frametitle{Почему не используется}
  \begin{itemize}
  \item{ Компилятор в язык реализации интерпретатора
      $$Interpreter^{S}_{\mathbf{L}}
      \xrightarrow{\qquad spec^{\mathbf{L}}_{\mathbf{L}}\qquad} Compiler^{S \to \mathbf{L}}_{\mathbf{L}}$$
      \hfill {\large \color{red}Это основная проблема}
    }
    \vfill
  \item {Апостериорный факт: реализовывать специализаторы сложно ($\sim$ как компиляторы)}
  \end{itemize}
\end{frame}

\begin{frame}\frametitle{Идея}
  \begin{itemize}
  \item {Специализатор для машинного кода
      $$Interpreter^{S}_{\mathbf{ASM}}
      \xrightarrow{\qquad spec^{ \mathbf{ASM}}_{\mathbf{ASM}}\qquad} Compiler^{S \to \mathbf{ASM}}$$
    }
    \vfill
  \item {Как получить $Interpreter^{S}_{\mathbf{ASM}}$?
      $$\llbracket gcc \rrbracket [Interpreter^{S}_{\textsc{C}}] =
      Interpreter^{S}_{\mathbf{ASM}}$$}
    \vfill
  \item {Как получить $spec^{\mathbf{ASM}}_{\mathbf{ASM}}$?
      $$\llbracket gcc \rrbracket [spec^{ \mathbf{ASM}}_{\textsc{C}}] =
      spec^{\mathbf{ASM}}_{\mathbf{ASM}}$$}
    \vfill
  \item{Как получить $spec^{\mathbf{ASM}}_{\textsc{C}}$?}
    \vfill
  \end{itemize}
\end{frame}

\begin{frame}\frametitle{Цель и задачи}
\begin{block}{Цель}
Исследование возможности и особенностей специализации машинного код
\end{block}
\begin{block}{Задачи}
\footnotesize{
\begin{itemize}
\item Изучить существующие подходы и алгоритмы специализации низкоуровневых языков программирования
\item Исследовать особенности специализации машинного кода
\item Предложить алгоритм специализации машинного кода
\item Реализовать прототип специализатора для машинного кода на языке C, основанного на предложенном алгоритме специализации
\item Произвести апробацию возможностей полученного специализатора
\end{itemize}
}
\end{block}
\end{frame}

\begin{frame}{Существующие подходы}
	Partial evaluation and automatic program generation, 
	Neil D. Jones, Carsten K. Gomard, Peter Sestoft, 1994. 
    \begin{itemize}
    \vfill
    \item Специализатор для языка Flow Chart
    \vfill
    \item Flow Chart структурно похож на машинный код
    \vfill
    \item Специализатор является самоприменимым
  	\end{itemize}
  \vfill
    Partial evaluation of machine code,
    Srinivasan Venkatesh, Reps Thomas, 2015.
    \vfill
    \begin{itemize}
    \item Специализатор для подмножества IA-32
    \vfill
    \item Использованы сложные техники
    \vfill
    \item Реализован на Java $\Rightarrow$ нельзя самоприменить
  	\end{itemize}
\end{frame}


\lstset{language=C}
\begin{frame}[fragile]\frametitle{Binding time analisys(BTA)}
Анализ времени исполнения классифицирует переменные/инструкции на \color{ao}{статические} \color{black} и \color{red}{динамические}
\vfill
\color{black}
\begin{columns}
\begin{column}{0.4\textwidth}
\begin{lstlisting}[
  belowcaptionskip=.2\baselineskip,
  breaklines=true,
  xleftmargin=\parindent,
  showstringspaces=false,
  basicstyle=\fontsize{8}{4}\selectfont\ttfamily,
  keywordstyle=\bfseries,
  commentstyle=\itshape\color{gray!40!black},
  stringstyle=\color{ao},
  numbers=left,
  numbersep=5pt,
  numberstyle=\fontsize{6}{4}\selectfont\ttfamily\color{gray},]
int pow(@\textcolor{red}{int a}@, @\textcolor{ao}{int b}@) {
    int res = 1;
    while (b > 0) {
        res = res * a;
        b = b - 1;
    }
    return a;
}
\end{lstlisting}
\end{column}
\begin{column}{0.2\textwidth}
$$\xrightarrow{\quad BTA \quad}$$
\end{column}
\begin{column}{0.4\textwidth}
\begin{lstlisting}[
  belowcaptionskip=.2\baselineskip,
  breaklines=true,
  xleftmargin=\parindent,
  showstringspaces=false,
  basicstyle=\fontsize{8}{4}\selectfont\ttfamily,
  keywordstyle=\bfseries,
  commentstyle=\itshape\color{gray!40!black},
  stringstyle=\color{ao},
  numbers=left,
  numbersep=5pt,
  numberstyle=\fontsize{6}{4}\selectfont\ttfamily\color{gray},]
int pow(@\textcolor{red}{int a}@, @\textcolor{ao}{int b}@) {
    @\textcolor{red}{int res = 1;}@
    while (@\textcolor{ao}{b > 0}@) {
        @\textcolor{red}{res = res * a}@;
        @\textcolor{ao}{b = b - 1}@;
    }
    @\textcolor{red}{return a}@;
}
\end{lstlisting}
\end{column}
\end{columns}
\end{frame}

\lstset{language=C}
\begin{frame}[fragile]\frametitle{Особенности специализации машинного кода}
Проблема : BTA делает регистры динамическими \\
\begin{lstlisting}[
  belowcaptionskip=.2\baselineskip,
  breaklines=true,
  xleftmargin=\parindent,
  showstringspaces=false,
  basicstyle=\fontsize{9}{4}\selectfont\ttfamily,
  keywordstyle=\bfseries,
  commentstyle=\itshape\color{gray!40!black},
  stringstyle=\color{ao},
  numbers=left,
  numbersep=5pt,
  numberstyle=\fontsize{9}{4}\selectfont\ttfamily\color{gray},]
//%esi динамический
mov %esi %eax
//теперь %eax динамический
mov 4 %eax
//@специализатор   не   знает   значение@ %eax
\end{lstlisting}
Решение : Online специализация \\
\vfill
Проблема : Комплексные инструкции (push)\\
\begin{lstlisting}[
  belowcaptionskip=.2\baselineskip,
  breaklines=true,
  xleftmargin=\parindent,
  showstringspaces=false,
  basicstyle=\fontsize{9}{4}\selectfont\ttfamily,
  keywordstyle=\bfseries,
  commentstyle=\itshape\color{gray!40!black},
  stringstyle=\color{ao},
  numbers=left,
  numbersep=5pt,
  numberstyle=\fontsize{9}{4}\selectfont\ttfamily\color{gray},]
//%esi динамический
push %esi
//теперь %esp динамический
push 4
pop %eax
//@специализатор   не   знает   значение@ %eax
\end{lstlisting}
Решение : BTA разделяет инструкцию на простые\\
\end{frame}


\lstset{language=C}
\begin{frame}[fragile]\frametitle{Особенности машинного кода}
Проблема : Специализация констант времени исполнения \\
\begin{lstlisting}[
  belowcaptionskip=.2\baselineskip,
  breaklines=true,
  xleftmargin=\parindent,
  showstringspaces=false,
  basicstyle=\fontsize{9}{4}\selectfont\ttfamily,
  keywordstyle=\bfseries,
  commentstyle=\itshape\color{gray!40!black},
  stringstyle=\color{ao},
  numbers=left,
  numbersep=5pt,
  numberstyle=\fontsize{9}{4}\selectfont\ttfamily\color{gray},]
//%esi динамический
push %esi
//@специализируется в@
mov %esi (268123094)
//@адрес может меняться между запусками@ %eax
\end{lstlisting}
\vfill
Решение : Символьные вычисления
\begin{lstlisting}[
  belowcaptionskip=.2\baselineskip,
  breaklines=true,
  xleftmargin=\parindent,
  showstringspaces=false,
  basicstyle=\fontsize{9}{4}\selectfont\ttfamily,
  keywordstyle=\bfseries,
  commentstyle=\itshape\color{gray!40!black},
  stringstyle=\color{ao},
  numbers=left,
  numbersep=5pt,
  numberstyle=\fontsize{9}{4}\selectfont\ttfamily\color{gray},]
//%esi динамический
push %esi
//@специализируется в@
mov %esi -48(0)
\end{lstlisting}
\end{frame}

\lstset{language=C}
\begin{frame}[fragile]{Алгоритм}
\begin{lstlisting}[
  belowcaptionskip=.2\baselineskip,
  breaklines=true,
  xleftmargin=\parindent,
  showstringspaces=false,
  basicstyle=\fontsize{9}{4}\selectfont\ttfamily,
  keywordstyle=\bfseries,
  commentstyle=\itshape\color{gray!40!black},
  stringstyle=\color{ao},
  numbers=left,
  numbersep=5pt,
  numberstyle=\fontsize{9}{4}\selectfont\ttfamily\color{gray},]
Input: program, program_input
state @$\leftarrow$@ generate_state(program, program_input)
queue @$\leftarrow$@ state
while(!queue.empty()) {
    st @$\leftarrow$@ queue.pop()
    while(!st.end()){
        cmd @$\leftarrow$@ st.read_cmd()
        params @$\leftarrow$@ st.read_params()
        if (BTA(params) == static) {
            st @$\leftarrow$@ st.eval(cmd, params)
        }
        else {
            st @$\leftarrow$@ st.dynamic_eval(cmd, params)
            reduced @$\leftarrow$@ st.reduce(cmd, params)
            result @$\leftarrow$@ result ++ reduced
        }
    }
}
Output: result
\end{lstlisting}
\end{frame}

\begin{frame}{Реализация прототипа}
\begin{itemize}
\vfill
\item Реализован на языке С
\vfill
\item Выбран набор ключевых тестов и реализован фукционал для прохождения данных тестов
\vfill
\item Специализатор способен обрабатывать основные машинные инструкции: mov, lea, add, sub, imul, cmp, test, call, ret, push, pop, jmp, jcc
\vfill
\item Результатом работы специализатора является программа на языке, схожим с ассемблером. Язык позволяет наглядно представлять результаты специализации 
\vfill
\item Генерация машинного кода реализована для узкого набора инcтрукций
\vfill
\end{itemize}
\end{frame}

\lstset{language=C}
\begin{frame}[fragile]\frametitle{KMP тест }
\begin{itemize}
\item $\llbracket spec \rrbracket_{ASM}[kmp, "ababac"]$
\vfill
\item Для прохождения данного теста необходимо преобразовать наивный алгоритм поиска подстроки в строке в оптимальный (например, KMP).
\vfill
\item Искомая строка является \color{ao} статической \color{black}
\vfill
\item Строка, в которой происходит поиск, является \color{red} динамической \color{black}
\vfill
\item Тест пройден 
\end{itemize}

\end{frame}

\begin{frame}[fragile]\frametitle{Специализация интерпретатора}
\begin{tabular}{l  c  r}

\begin{minipage}[t]{0.3\textwidth}
\footnotesize{
int add(\textcolor{blue}{int a, int b})\{\\
\hspace*{10mm} int c;\\
\hspace*{10mm} \textcolor{br}{c = a};\\
\hspace*{10mm} \textcolor{ao}{c += b};\\
\hspace*{10mm} return c;\\
\}\\
}
\end{minipage}
& 
\begin{minipage}[t]{0.4\textwidth}
$$\xrightarrow{ \llbracket spec \rrbracket_{ASM}[interpreter, add] }$$
\end{minipage}    & 
\multirow{2}{*}{
\begin{minipage}[c]{0.4\textwidth}
\tiny{
Start block -697046\\
mov89 \%rsi  -72(0) \\
call malloc\\
\textcolor{blue}{mov8b -72(0)  \%rax}\\
\textcolor{blue}{mov8b 0(rax)  \%rax}\\
\textcolor{blue}{mov89 \%rax  0(2)}\\
\textcolor{blue}{add83 -72(0)  4}\\
\textcolor{blue}{mov8b -72(0)  \%rax}\\
\textcolor{blue}{mov8b 0(rax)  \%rax}\\
\textcolor{blue}{mov89 \%rax  16(2)}\\
\textcolor{blue}{add83 -72(0)  4}\\
\textcolor{br}{mov8b 0(2)  \%rax}\\ 
\textcolor{red}{mov89 \%rax  -100(0)}\\ 
\textcolor{red}{mov8b -100(0)  \%rax}\\ 
\textcolor{br}{mov89 \%rax  -28(0)}\\ 
\textcolor{br}{mov8b -28(0)  \%rax}\\ 
\textcolor{br}{mov89 \%rax  32(2)}\\
\textcolor{ao}{mov8b 16(2)  \%rax}\\ 
\textcolor{red}{mov89 \%rax  -100(0)}\\ 
\textcolor{red}{mov8b -100(0)  \%rax}\\ 
\textcolor{ao}{mov89 \%rax  -28(0)}\\ 
\textcolor{ao}{mov8b 32(2)  \%rcx}\\ 
\textcolor{ao}{mov8b -28(0)  \%rdx}\\ 
\textcolor{ao}{add01 \%rcx  \%rdx}\\ 
\textcolor{ao}{mov89 \%rdx  32(2)}\\ 
mov8b 32(2)  \%rax \\
mov89 \%rax  -100(0)\\ 
mov8b -100(0)  \%rax \\
ret\\
}
\vspace*{20mm}
\end{minipage}
}\\
 \multicolumn{2}{c}{
\begin{minipage}[c]{0.7\textwidth}
\begin{itemize}
\vfill
\item Цель данного теста - 
оценить количество лишних инструкций
\vspace*{5mm}
\end{itemize}
\end{minipage}}
 & \\
\multicolumn{2}{c}{
\begin{minipage}[c]{0.7\textwidth}
\begin{itemize}
\vfill
\item Результат специализации отвечает требуемой функциональности. Лишние инструкции присутствуют
\end{itemize}
\end{minipage}}  & \\
\end{tabular}
\vfill
\end{frame}


\begin{frame}\frametitle{Итоги}
\begin{itemize}
\item Исследованы различные подходы специализации низкоуровневых языков
\vfill
\item Исследованы особенности специализации машинного кода
\vfill
\item Предложен алгоритм специализации машинного кода
\vfill
\item Реализован прототип специализатора на языке С и произведена его апробация
\vfill
\end{itemize}
\end{frame}
\end{document}